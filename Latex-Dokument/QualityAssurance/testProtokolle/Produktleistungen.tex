\begin{table}[]
\caption{Testfall 8.8 (Testfälle für die Datenverwaltung)}
\centering
	\begin{tabular}{| p{0.15\linewidth} | p{0.15\linewidth} | p{0.20\linewidth} |
	p{0.15\linewidth} | p{0.1\linewidth} | p{0.1\linewidth} |}
	\hline
	\textbf{Testfall} &
	\textbf{Abgedeckte Funktionalitäten} &
	\textbf{Beschreibung} &
	\textbf{Ergebnis} & \textbf{Lukas}
	(Windows 10) Version 1.4.22 &
	\textbf{Niels} (Linux Mint Cinnamon 3.0.7) Version 1.4.22 
\\
\hline
 &
/NF10/ &
Man startet das Programm normal und öffnet den CEditor. Hier tippt man nun in
die Mitte der voting Methode und schreibt "`for"'. Nun drückt man "`strg +
leer"'. & Nach weniger als 0.5 Sekunden öffnet sich ein Fenster, welches alle
Autovervollständigungen anzeigt & \Checkmark &
\Checkmark
\\
\hline
 &
/NF30/ &
Man startet das Programm normal und öffnet den CEditor. Hier tippt man nun 10000
Zeilen richtigen C-Code ein (Beispielsweise 1 Zeile: int i = 1; 1000
Zeilen: i++; Am Ende: return i;). Im
Paramtereditor stellt man alle Parameter moderat ein (alles unter 10, Timeout
ausgestellt). Im Eigenschafteneditor läd man die Eigenschaft
"`FalseProperty.props"' und startet die Analyse. & Nach einiger Zeit schließt
die Analys ab und man kann das Ergebnis angucken & \Checkmark &
\Checkmark
\\
\hline
 &
/NF20/ /NF40/ /NF50/ /NF60/ &
Man startet das Programm normal und öffnet den CEditor. Hier tippt man nun 10000
Zeilen richtigen C-Code ein (Beispielsweise 1 Zeile: int i = 1; 1000
Zeilen: i++; Am Ende: return i;).
Im Paramtereditor stellt man alle Parameter auf 10000 ein und den TimeOut auf
15 Minuten (der TimeOut ist sehr linear, wenn er 15 Minuten schafft, schafft er
auch mehrere Tage / Jahre).
Im Eigenschafteneditor läd man die Eigenschaft "`FalseProperty.props"' und
startet die Analyse. & Nach ziemlich genau 15 Minuten hört die Überprüfung auf,
und die Eigenschaft wird als durch einen Timeout abgebrochen angezeigt &
\Checkmark &
\Checkmark
\\
\hline 

\end{tabular}
\end{table}