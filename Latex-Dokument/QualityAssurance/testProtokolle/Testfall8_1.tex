\begin{table}[]
\caption{Testfall 8.1 (Testfälle für die Datenverwaltung)}
\centering
	\begin{tabular}{| p{0.10\linewidth} | p{0.15\linewidth} | p{0.27\linewidth} |
	p{0.15\linewidth} | p{0.09\linewidth} | p{0.09\linewidth} |}
	\hline
	\textbf{Sub-Testfall} &
	\textbf{Abgedeckte Funktionalitäten} &
	\textbf{Beschreibung} &
	\textbf{Ergebnis} & \textbf{Lukas}
	(Windows 10) Version 1.4.22 &
	\textbf{Justin} Lubuntu 16.1 Version 1.4.19
\\
\hline
/T010/ (C-Editor) &
/FS1030/ /FS1100/ /FS1110/ & 
Man gibt ein Wahlverfahren ein. Man wählt in der Toolbar den Button "`Neu"' aus. In einen Dialog gibt man das gewünschte Wahlverfahren und die Anzahl der Sitze ein. In ein Textfeld wird der Name eingegeben. Man drückt auf den Button "`Erstellen"'.
 &
Ein neuer vorgefertigter C-Code erscheint im C-Editor. Ausgegraut sind die Argumente des Wahlverfahrens. &
\Checkmark & \Checkmark
\\
\hline 
/T010/ (Eigenschafteneditor) &
/FM2100/ /FS2150/ &
Man gibt formale Eigenschaften ein. Man wählt in der Toolbar den Button "`Neu"' aus. 
 &
Die Felder für "`Symbolische Variablen"', "`Vorbedingungen"' und "`Nachbedingungen"' leeren sich. In der Titelleiste erscheint der Name "`Eigenschaft 0"'. &
\centering \Checkmark & \Checkmark
\\
\hline 
/T010/ (Eigenschaftenliste) &
/FM3020/ &
Man fügt Eigenschaften zur Liste hinzu. Man wählt in der Toolbar den Button "`Neu"' aus. Die Nachfrage, ob man speichern will, wird verneint.
 &
Die Liste der Eigenschaften leert sich. &
\Checkmark & \Checkmark
\\
\hline 
/T010/ (Parametereditor) &
/FM4050/ &
Man ändert die Parameter. Man wählt in der Toolbar den Button "`Neu"' aus.
 &
Es existiert kein Button zum Erstellen eines neuen Projekts. &
X & X
\\
\hline



\end{tabular}
\end{table}