\begin{table}[]
\caption{Testfall 8.2 (Testfall für Rückgängig machen und Wiederherstellen)}
\centering
	\begin{tabular}{| p{0.15\linewidth} | p{0.15\linewidth} | p{0.20\linewidth} |
	p{0.15\linewidth} | p{0.1\linewidth} | p{0.1\linewidth} |}
	\hline
	\textbf{Sub-Testfall} &
	\textbf{Abgedeckte Funktionalitäten} &
	\textbf{Beschreibung} &
	\textbf{Ergebnis} & \textbf{Lukas}
	(Windows 10) Version ??? &
	\textbf{Justin} Lubuntu 16.1 Version 1.4.19) 
\\
\hline
/T010/ (C-Editor) &
/FS1030/ /FS1100/ /FS1110/ &
Man gibt ein Wahlverfahren ein. Man wählt in der Toolbar den Button "`Neu"' aus. In einen Dialog gibt man das gewünschte Wahlverfahren und die Anzahl der Sitze ein. In ein Textfeld wird der Name eingegeben. Man drückt auf den Button "`Erstellen"'.
 &
Ein neuer vorgefertigter C-Code erscheint im C-Editor. Ausgegraut sind die Argumente des Wahlverfahrens. &
\centering . & \Checkmark
\\
\hline 
/T010/ (Eigenschafteneditor) &
/FM2100/ /FS2150/ &
Man gibt formale Eigenschaften ein. Man wählt in der Toolbar den Button "`Neu"' aus. 
 &
Die Felder für "`Symbolische Variablen"', "`Vorbedingungen"' und "`Nachbedingungen"' leeren sich. In der Titelleiste erscheint der Name "`Eigenschaft 0"'. &
\centering , & \Checkmark
\\
\hline 
/T010/ (Eigenschaftenliste) &
/FM3020/ &
Man fügt Eigenschaften zur Liste hinzu. Man wählt in der Toolbar den Button "`Neu"' aus. Die Nachfrage, ob man speichern will, wird verneint.
 &
Die Liste der Eigenschaften leert sich. &
\centering . & \Checkmark
\\
\hline 
/T010/ (Parametereditor) &
/FM4050/ &
Man ändert die Parameter. Man wählt in der Toolbar den Button "`Neu"' aus.
 &
Es existiert kein Button für das Neu erstellen. &
\centering . & X
\\
\hline
/T020/ /T030/ (C-Editor) &
/FM1030/ /FS1100/ /FS1040/ /FS1060/ &
Man gibt ein Wahlverfahren ein. Man wählt in der Toolbar den Button "`Speichern"' aus. In einen Dialog gibt man den gewünschten Speicherort ein. Man drückt auf den Button "`Speichern"'. Man wählt in der Toolbar den Button "`Öffnen"' aus. In einem Dialog wählt man das gespeicherte Wahlverfahren aus.
 &
Das Wahlverfahren wurde gespeichert. Das Laden des Wahlverfahrens schlug fehl. Das Format wurde nicht erkannt. &
\centering . & X
\\
\hline 
/T020/ /T030/ (Eigenschafteneditor) &
/FM2100/ /FS2110/ &
Man gibt formale Eigenschaften ein. Man wählt in der Toolbar den Button "`Speichern"' aus. In einen Dialog gibt man den gewünschten Speicherort ein. Man drückt auf den Button "`Speichern"'. Man wählt in der Toolbar den Button "`Öffnen"' aus. In einem Dialog wählt man die gespeicherten formalen Eigenschaften aus.
 &
Die Eigenschaft wurde gespeichert. Das Laden schlägt fehl. Das Format wurde nicht erkannt. &
\centering . & X
\\
\hline 
/T020/ /T030/ (Eigenschaftenliste) &
/FM3060/ /FM3070/ &
Man fügt Eigenschaften zur Liste hinzu. Man wählt in der Toolbar den Button "`Speichern"' aus. In einen Dialog gibt man den gewünschten Speicherort ein. Man drückt auf den Button "`Speichern"'. Man wählt in der Toolbar den Button "`Öffnen"' aus. In einem Dialog wählt man die gespeicherte Eigenschaftenliste aus.
 &
Die Liste der Eigenschaften wurde gespeichert. Die Liste wird wieder geladen. &
\centering . & \Checkmark
\\
\hline 
/T020/ /T030/ (Parametereditor) &
/FM4050/ /FM4060/ &
Man ändert die Parameter. Man wählt in der Toolbar den Button "`Speichern"' aus. In einen Dialog gibt man den gewünschten Speicherort ein. Man drückt auf den Button "`Speichern"'. Man wählt in der Toolbar den Button "`Öffnen"' aus. In einem Dialog wählt man die gespeicherte Eigenschaftenliste aus.
 &
Das Projekt wird gespeichert. Das Projekt kann wieder geladen werden. &
\centering . & \Checkmark
\\




\hline 
/T020/ (Parametereditor) &
/FM4050/ &
Man ändert die Parameter. Man wählt in der Toolbar den Button "`Neu"' aus.
 &
Fehlschlag: Es existiert kein Button für das Neu erstellen. &
\centering . & X
\\
\hline 
/T010/& /FS1100/ /FS2150/ /F0010/ /F0050 &
Man startet das Programm ganz normal. Nun gibt man in
jedes Feld, das die "`Rückgängig machen"' Funktionalität unterstützt, einen
kleinen Text ein, und drückt dann, während der Fokus auf dem zu testendem Feld
liegt "`Strg + z"'. Nun drückt man "`Strg + r" &
Der vorher durch das rückgängig machen verschwundene Buchstabe oder Textblock
erscheint wieder &
\centering \Checkmark 
& X 
\\ \hline

\end{tabular}
\end{table}