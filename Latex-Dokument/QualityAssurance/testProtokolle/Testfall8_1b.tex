\begin{table}[]
\caption{Testfall 8.1 (Testfälle für die Datenverwaltung)}
\centering
	\begin{tabular}{| p{0.15\linewidth} | p{0.15\linewidth} | p{0.20\linewidth} |
	p{0.15\linewidth} | p{0.1\linewidth} | p{0.1\linewidth} |}
	\hline
	\textbf{Sub-Testfall} &
	\textbf{Abgedeckte Funktionalitäten} &
	\textbf{Beschreibung} &
	\textbf{Ergebnis} & \textbf{Lukas}
	(Windows 10) Version ??? &
	\textbf{Justin} Lubuntu 16.1 Version 1.4.19) 
\\
\hline
/T020/ /T030/ (C-Editor) &
/FM1030/ /FS1100/ /FS1040/ /FS1060/ &
Man gibt ein Wahlverfahren ein. Man wählt in der Toolbar den Button "`Speichern"' aus. In einen Dialog gibt man den gewünschten Speicherort ein. Man drückt auf den Button "`Speichern"'. Man wählt in der Toolbar den Button "`Öffnen"' aus. In einem Dialog wählt man das gespeicherte Wahlverfahren aus.
 &
Das Wahlverfahren wurde gespeichert. Das Laden des Wahlverfahrens schlug fehl. Das Format wurde nicht erkannt. &
\centering . & X
\\
\hline 
/T020/ /T030/ (Eigenschafteneditor) &
/FM2100/ /FS2110/ &
Man gibt formale Eigenschaften ein. Man wählt in der Toolbar den Button "`Speichern"' aus. In einen Dialog gibt man den gewünschten Speicherort ein. Man drückt auf den Button "`Speichern"'. Man wählt in der Toolbar den Button "`Öffnen"' aus. In einem Dialog wählt man die gespeicherten formalen Eigenschaften aus.
 &
Die Eigenschaft wurde gespeichert. Das Laden schlägt fehl. Das Format wurde nicht erkannt. &
\centering . & X
\\
\hline 


\end{tabular}
\end{table}