\begin{table}[]
\caption{Testfall 8.1 (Testfälle für die Datenverwaltung)}
\centering
	\begin{tabular}{| p{0.10\linewidth} | p{0.15\linewidth} | p{0.27\linewidth} |
	p{0.15\linewidth} | p{0.09\linewidth} | p{0.09\linewidth} |}
	\hline
	\textbf{Sub-Testfall} &
	\textbf{Abgedeckte Funktionalitäten} &
	\textbf{Beschreibung} &
	\textbf{Ergebnis} & \textbf{Lukas}
	(Windows 10) Version 1.4.22 &
	\textbf{Justin} Lubuntu 16.1 Version 1.4.19 
\\
\hline 
/T020/ /T030/ (Eigenschaftenliste) &
/FM3060/ /FM3070/ &
Man fügt Eigenschaften zur Liste hinzu. Man wählt in der Toolbar den Button "`Speichern"' aus. In einen Dialog gibt man den gewünschten Speicherort ein. Man drückt auf den Button "`Speichern"'. Man wählt in der Toolbar den Button "`Öffnen"' aus. In einem Dialog wählt man die gespeicherte Eigenschaftenliste aus.
 &
Die Liste der Eigenschaften wurde gespeichert. Die Liste wird wieder geladen. &
\Checkmark & \Checkmark
\\
\hline 
/T020/ /T030/ (Parametereditor) &
/FM4050/ /FM4060/ &
Man ändert die Parameter. Man wählt in der Toolbar den Button "`Speichern"' aus. In einen Dialog gibt man den gewünschten Speicherort ein. Man drückt auf den Button "`Speichern"'. Man wählt in der Toolbar den Button "`Öffnen"' aus. In einem Dialog wählt man die gespeicherte Eigenschaftenliste aus.
 &
Das Projekt wird gespeichert. Das Projekt kann wieder geladen werden. &
\Checkmark & \Checkmark
\\ \hline 

\end{tabular}
\end{table}