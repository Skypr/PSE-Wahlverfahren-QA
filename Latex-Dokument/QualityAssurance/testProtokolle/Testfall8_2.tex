\begin{table}[]
\caption{Testfall 8.2 (Testfall für Rückgängig machen und Wiederherstellen)}
\centering
	\begin{tabular}{| p{0.15\linewidth} | p{0.15\linewidth} | p{0.20\linewidth} |
	p{0.15\linewidth} | p{0.1\linewidth} | p{0.1\linewidth} |}
	\hline
	\textbf{Sub-Testfall} &
	\textbf{Abgedeckte Funktionalitäten} &
	\textbf{Beschreibung} &
	\textbf{Ergebnis} & \textbf{Lukas}
	(Windows 10) Version 1.4.13 &
	\textbf{Niels} (Linux Mint Cinnamon 3.0.7) Version 1.4.22 
\\
\hline
/T100/ &
/FS1100/ /FS2150/ /F0010/ /F0050/ &
Man startet das Programm ganz normal. Nun gibt man in
jedes Feld, das die "`Rückgängig machen"' Funktionalität unterstützt, einen
kleinen Text ein, und drückt dann, während der Fokus auf dem zu testendem Feld
liegt "`Strg + z"' &
Der zuletzt eingegebene Buchstabe
oder Textblock (im Falle des Einfügens mit "`Strg + c"') wird gelöscht &
\Checkmark & \Checkmark 
\\
\hline /T110/& /FS1100/ /FS2150/ /F0010/ /F0050 &
Man startet das Programm ganz normal. Nun gibt man in
jedes Feld, das die "`Rückgängig machen"' Funktionalität unterstützt, einen
kleinen Text ein, und drückt dann, während der Fokus auf dem zu testendem Feld
liegt "`Strg + z"'. Nun drückt man "`Strg + r" &
Der vorher durch das rückgängig machen verschwundene Buchstabe oder Textblock
erscheint wieder &
\Checkmark 
& \Checkmark  
\\ \hline

\end{tabular}
\end{table}