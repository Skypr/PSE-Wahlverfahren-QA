

\begin{table}[]
\caption{Testfall 8.3 (Testfall für Kopieren, Einfügen und Ausschneiden in den Editoren)}
\centering
	\begin{tabular}{| p{0.09\linewidth} | p{0.14\linewidth} | p{0.27\linewidth} |
	p{0.15\linewidth} | p{0.1\linewidth} | p{0.1\linewidth} |}
	\hline
	\textbf{Sub-Testfall} &
	\textbf{Abgedeckte Funktionalitäten} &
	\textbf{Beschreibung} &
	\textbf{Ergebnis} & \textbf{Niels}
	(Windows 10) Version 1.4.22 &
	\textbf{Niels} (Linux Mint Cinnamon 3.0.7) Version 1.4.22
\\
\hline
/T200/ &
/F0010/ /FS1100/ /FS2150/ &
Man startet das Programm ganz normal. Nun öffnet man den den jeweiligen Editor (C-Editor und Eigenschafteneditor) und gibt einen kleinen Text ein.
Man markiert den kleinen Text und drückt dann, während der Fokus auf dem Editor liegt, "`Strg + x"' oder betätigt den Button für Ausschneiden. 
&
Der markierte Text wird gelöscht und in den Zwischenspeicher gespeichert. &
\Checkmark & \Checkmark
\\
\hline
/T200/ &
/F0010/ /FS1100/ /FS2150/ &
Man startet das Programm ganz normal. Nun öffnet man den den jeweiligen Editor (C-Editor und Eigenschafteneditor) und gibt einen kleinen Text ein.
Man markiert den kleinen Text und drückt dann, während der Fokus auf dem Editor liegt, "`Strg + c"' oder betätigt den Button für Kopieren
&
Der markierte Text wird in den Zwischenspeicher gespeichert. &
\Checkmark & \Checkmark
\\
\hline
/T200/ &
/F0010/ /FS1100/ /FS2150/ &
Man startet das Programm ganz normal. Nun öffnet man den den jeweiligen Editor (C-Editor und Eigenschafteneditor)
Man drückt "`Strg + c"' oder betätigt den Button für Einfügen.
&
Falls ein Text im Zwischenspeicher gespeichert ist, wird er im Editor eingefügt. &
\Checkmark & \Checkmark
\\
\hline

\end{tabular}
\end{table}