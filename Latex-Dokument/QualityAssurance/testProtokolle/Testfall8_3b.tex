\begin{table}[]
\caption{Testfall 7 (Nichtfunktionale Anforderungen)}
\centering
	\begin{tabular}{| p{0.09\linewidth} | p{0.14\linewidth} | p{0.21\linewidth} |
	p{0.21\linewidth} | p{0.1\linewidth} | p{0.1\linewidth} |}
	\hline
	\textbf{Sub-Testfall} &
	\textbf{Abgedeckte Funktionalitäten} &
	\textbf{Beschreibung} &
	\textbf{Ergebnis} & \textbf{Lukas}
	(Windows 10) Version 1.4.22 &
	\textbf{Niels} (Linux Mint Cinnamon 3.0.7) Version 1.4.22
\\
\hline
 &
/NF10/ &
Man startet das Programm ganz normal. Nun öffnet man den C-Editor und gibt in
die Mitte der Voting Methode "`for"' ein. Nun drückt man "`strg"' +
"`leer"' & In weniger als 0.5 Sekunden öffnet sich ein Fenster, welches die
Code-Completion anzeigt. & \Checkmark & \Checkmark
\\
\hline
 &
/NF30/ &
Man startet das Programm ganz normal und öffnet den C-Editor. Hier gibt man nun
einen Code ein, der über 10000 Zeilen lang ist. Im Parametereditor wählt man
alle Variablen kleiner als 10 und stellt den Timeout aus. Im Eigenschafteneditor
öffnet man "`FalseProperty.props"'. Nun startet man die Überprüfung & Nach
kurzer Zeit beendet sich die Überprüfung, und man kann das Ergebnis im
Eigenschafteneditor ablesen. & \Checkmark &
\Checkmark
\\
\hline
 &
/NF20/ /NF40/ /NF50/ /NF60/ &
Man startet das Programm ganz normal und öffnet den C-Editor. Hier gibt man nur
einen sehr einfachen Code ein. Im Eigenschafteneditor erstellt man eine neue
Eigenschaft, welche 10 Vor- und Nachbedingungen enthält. Im Parametereditor
stellt man alle Werte auf 10000, und den Timeout auf 15 Minuten & CBMC startet
mit der Überprüfung der Eigenschaft. Nach 15 Minuten hört es mit der Überprüfung
auf. & \Checkmark &
\Checkmark
\\
\hline
\end{tabular}
\end{table}
