\begin{table}[]
\caption{Testfall 8.3 (Bearbeiten des Codes in den Editoren)}
\centering
	\begin{tabular}{| p{0.09\linewidth} | p{0.14\linewidth} | p{0.21\linewidth} |
	p{0.21\linewidth} | p{0.1\linewidth} | p{0.1\linewidth} |}
	\hline
	\textbf{Sub-Testfall} &
	\textbf{Abgedeckte Funktionalitäten} &
	\textbf{Beschreibung} &
	\textbf{Ergebnis} & \textbf{Niels}
	(Windows 10) Version 1.4.22 &
	\textbf{Niels} (Linux Mint Cinnamon 3.0.7) Version 1.4.22
\\
\hline
/T210/ &
/FS1130/ /FK2140/ &
Man startet das Programm ganz normal. Nun öffnet man den den jeweiligen Editor (CEditor und Eigenschafteneditor) und gibt ein Wort innerhalb des Editors teilweise ein, dass der richtigen Syntax entspricht. Nun betätigt man "`Strg + Leertaste"'
&
Es öffnet sich ein Fenster für die Autocompletion, aus der man die gewünschte Eingabe wählen kann, welche nach Auswahl hinzugefügt wird.   &
\Checkmark & \Checkmark
\\

\hline
/T210/ &
/FS1120/ /FS1130/ &
Man startet das Programm ganz normal. Man öffnet den C-Editor. Man gibt C-Code ein. 
&
Klammern und Anführungszeichen werden automatisch geschlossen.
Code in Schleifen und if-Statements wird automatisch eingerückt. &
\Checkmark & \Checkmark
\\
\hline

\end{tabular}
\end{table}
