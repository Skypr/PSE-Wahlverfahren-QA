\begin{table}[]
\caption{Testfälle 8.4 (Testfälle für den C-Editor)}
\centering
	\begin{tabular}{| p{0.15\linewidth} | p{0.15\linewidth} | p{0.20\linewidth} |
	p{0.15\linewidth} | p{0.1\linewidth} | p{0.1\linewidth} |}
	\hline
	\textbf{Sub-Testfall} &
	\textbf{Abgedeckte Funktionalitäten} &
	\textbf{Beschreibung} &
	\textbf{Ergebnis} & \textbf{Holger} Windows 7  &
	\textbf{Holger} Ubuntu (16.04 LTS))
\\
\hline
/T310/ &
/FS1110/&
Man startet das Programm und öffnet den C-Editor. Dort wählt man eine der Möglichkeiten eine neue Wahlbeschreibung zu entwerfen
&
& 
&
\\
\hline
&
/FS1100/&
Per Shortcut: Strg + n
& Der Dialog zum Erstellen einer neuen Wahlverfahrensbeschreibung wird angezeigt
& \Checkmark
&\Checkmark
\\
\hline
&
&
Per Menü: Datei, dann Neu
& Der Dialog zum Erstellen einer neuen Wahlverfahrensbeschreibung wird angezeigt
& \Checkmark
&\Checkmark
\\
\hline
&
&
Per Toolbar: Erster Button der Toolbar
& Der Dialog zum Erstellen einer neuen Wahlverfahrensbeschreibung wird angezeigt
& \Checkmark
&\Checkmark
\\
\hline
\end{tabular}
\end{table}

\begin{table}[]
\caption{Testfälle 8.4 (Testfälle für den C-Editor)}
\centering
	\begin{tabular}{| p{0.15\linewidth} | p{0.15\linewidth} | p{0.20\linewidth} |
	p{0.15\linewidth} | p{0.1\linewidth} | p{0.1\linewidth} |}
	\hline
	\textbf{Sub-Testfall} &
	\textbf{Abgedeckte Funktionalitäten} &
	\textbf{Beschreibung} &
	\textbf{Ergebnis} & \textbf{Holger} Windows 7 &
	\textbf{Holger} Ubuntu (16.04 LTS))
\\
\hline
& /FS1110/
& Auswahl eines Input- und Resulttypen sowie eines Namens. Klicken des "Erstellen"-Buttons
& Der Funktionskörper wird entsprechend aktualisiert 
& 
&
\\
\hline
&
& Single-chice 
& Input: unsigned int votes[V]
& \Checkmark
& \Checkmark
\\
\hline
&
& Preference 
& Input: unsigned int votes[V][C]
& \Checkmark
& \Checkmark
\\
\hline
&
& Approval 
& Input: unsigned int votes[V][C]
& \Checkmark
& \Checkmark
\\
\hline
&
& Weighted Approval 
& Input: unsigned int votes[V][C]
& \Checkmark
& \Checkmark
\\
\hline
&
& Candidate or not determined
& Result: unsigned int 
& \Checkmark
& \Checkmark
\\
\hline
&
& Seats per party
& Result: unsigned int *
& \Checkmark
& \Checkmark
\\
\hline
\end{tabular}
\end{table}

\begin{table}[]
\caption{Testfälle 8.4 (Testfälle für den C-Editor)}
\centering
	\begin{tabular}{| p{0.15\linewidth} | p{0.15\linewidth} | p{0.20\linewidth} |
	p{0.15\linewidth} | p{0.1\linewidth} | p{0.1\linewidth} |}
	\hline
	\textbf{Sub-Testfall} &
	\textbf{Abgedeckte Funktionalitäten} &
	\textbf{Beschreibung} &
	\textbf{Ergebnis} & \textbf{Holger} Windows 7 &
	\textbf{Holger} Ubuntu (16.04 LTS))
\\
\hline
/T320/ &
/FM1050/&
Man startet das Programm und öffnet den C-Editor. Dort gibt man ein Programm ein welches mehrere Fehler enthält. Danach wählt man im Menü Code "statische Analyse" aus.
Fehler: Fehlendes return, Zugriff auf nicht deklarierte Variable, Verwendung nicht deklarierter Funktion, Funktionsaufruf mit falschen Parametern, fehlendes Semikolon, Fehlende schließende geschweifte Klammer nach for-Schleife
& Es werden alle Fehler im Code angezeigt
& \Checkmark
& \Checkmark
\\
\hline
\end{tabular}
\end{table}

\begin{table}[]
\caption{Testfälle 8.4 (Testfälle für den C-Editor)}
\centering
	\begin{tabular}{| p{0.15\linewidth} | p{0.15\linewidth} | p{0.20\linewidth} |
	p{0.15\linewidth} | p{0.1\linewidth} | p{0.1\linewidth} |}
	\hline
	\textbf{Sub-Testfall} &
	\textbf{Abgedeckte Funktionalitäten} &
	\textbf{Beschreibung} &
	\textbf{Ergebnis} & \textbf{Holger} Windows 7 &
	\textbf{Holger} Ubuntu (16.04 LTS))
\\
\hline
&
/FM1030/, /FM1040/&
Man startet das Programm und öffnet den C-Editor. Dort speichert man die geöffnete Wahlverfahrensbeschreibung an. einem beliebigen Ort. Danach klickt man auf Öffnen, navigiert an den Ort an dem die Datei gerade gespeichert wurde, und öffnet sie 
& Die gespeicherte Datei wird angezeigt
& \Checkmark
& \Checkmark
\\
\hline
\end{tabular}
\end{table}

\begin{table}[]
\caption{Testfälle 8.4 (Testfälle für den C-Editor)}
\centering
	\begin{tabular}{| p{0.15\linewidth} | p{0.15\linewidth} | p{0.20\linewidth} |
	p{0.15\linewidth} | p{0.1\linewidth} | p{0.1\linewidth} |}
	\hline
	\textbf{Sub-Testfall} &
	\textbf{Abgedeckte Funktionalitäten} &
	\textbf{Beschreibung} &
	\textbf{Ergebnis} & \textbf{Holger} Windows 7 &
	\textbf{Holger} Ubuntu (16.04 LTS))
\\
\hline
&
/FK1130/&
Man startet das Programm und öffnet den C-Editor. Dort geht man in den Körper der voting-Funktion und beginnt return zu tippen. Nach den ersten zwei Buchstaben betätigt man den Shortcut Strg - Leer. In dem erschienenen Menü wählt man return aus und drückt Enter.
& Das Wort return wird in den Funktionskörper geschrieben
& \Checkmark
& \Checkmark
\\
\hline
&
/FK1130/&
Man gibt in den Funktionskörper der voting-Funktion den text int asdasdasd ein und wartet 10 Sekunden. Danach geht man auf eine neue Zeile und tippt a. Dann betätigt man den Shortcut Strg-Leer
& In dem erschienenen Menü wird nun asdasdasd als Option angezeigt
& \Checkmark
& \Checkmark
\\
\hline
\end{tabular}
\end{table}

\begin{table}[]
\caption{Testfälle 8.4 (Testfälle für den C-Editor)}
\centering
	\begin{tabular}{| p{0.15\linewidth} | p{0.15\linewidth} | p{0.20\linewidth} |
	p{0.15\linewidth} | p{0.1\linewidth} | p{0.1\linewidth} |}
	\hline
	\textbf{Sub-Testfall} &
	\textbf{Abgedeckte Funktionalitäten} &
	\textbf{Beschreibung} &
	\textbf{Ergebnis} & \textbf{Holger} Windows 7 &
	\textbf{Holger} Ubuntu (16.04 LTS))
\\
\hline
&
/FK1140/&
Man startet das Programm und öffnet den C-Editor. Dort geht man auf den Menüpunkt Editor -> Eigenschaften. In dem erschienenen Dialog wählt man einen anderen Font und Schriftgröße aus
& Der verwendete Font und Schriftgröße werden zu der gewählten aktualisiert. Diese Änderung bleibt auch nach Neustart des Programmes
& \Checkmark
& \Checkmark
\\
\hline
\end{tabular}
\end{table}