\begin{table}[]
\caption{Testfall 8.5 (Testfall für das Erstellen einer Eigenschaft im Eigenschafteneditor)}
\centering
	\begin{tabular}{| p{0.15\linewidth} | p{0.15\linewidth} | p{0.20\linewidth} |
	p{0.15\linewidth} | p{0.1\linewidth} | p{0.1\linewidth} |}
	\hline
	\textbf{Sub-Testfall} &
	\textbf{Abgedeckte Funktionalitäten} &
	\textbf{Beschreibung} &
	\textbf{Ergebnis} & \textbf{Lukas}
	(Windows 10) Version 1.4.22 &
	\textbf{Nikolai} Arch Linux (4.10.3-1-ARCH))
\\
\hline
/T410/ &
/FM2040/ /FM2050/ /FM2070/ /FM2071/ /FM2072/ /FM2073/ /FM2080/ /FM2100/ /FM2120/&
Man startet das Programm ganz normal. Nun gibt man im Eigenschafteneditor in den Vorbedingungen 'VOTES1 == VOTES2;' und in den Nachbedingungen 'ELECT1 != ELECT2;' ein. Durch Auswählen von "Statische Fehlersuche" testet man die Eigenschaft auf Korrektheit und kann diese anschließend mit dem entsprechenden Menüpunkt oder Toolbar-Button speichern. &
Es wird 'Fehler: 0' im Fehlerfenster angezeigt und die Eigenschaft hat sich ohne Fehlermeldung speichern lassen. &
\Checkmark & \Checkmark
\\
\hline

\end{tabular}
\end{table}