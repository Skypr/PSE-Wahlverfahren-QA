\begin{table}[]
\caption{Testfall 8.6 (Testfälle für die Eigenschaftenliste)}
\centering
	\begin{tabular}{| p{0.10\linewidth} | p{0.15\linewidth} | p{0.27\linewidth} |
	p{0.15\linewidth} | p{0.09\linewidth} | p{0.09\linewidth} |}
	\hline
	\textbf{Sub-Testfall} &
	\textbf{Abgedeckte Funktionalitäten} &
	\textbf{Beschreibung} &
	\textbf{Ergebnis} & \textbf{Lukas}
	(Windows 10) Version ??? &
	\textbf{Justin} Lubuntu 16.1 Version 1.4.19) 
\\
\hline
/T510/ &
/FM0010/ /FM0020/ /FM0030/ /FM0031/ &
Man gibt ein einfaches Wahlverfahren ein, das eine gewählte Person zurückgibt. Man erstellt eine erste Eigenschaft, die erfüllt ist, und eine zweite Eigenschaft, die nicht erfüllt ist. Man wählt im Parametereditor den Start der Analyse in der Toolbar aus.
 &
Die erste Eigenschaft erscheint grün. Die zweite Eigenschaft erscheint rot. Beim Klick auf das Augensymbol der zweiten Eigenschaft öffnet sich ein Fenster mit einem Gegenbeispiel. &
\centering . & \Checkmark
\\
\hline 
/T520/ &
/FM3010/ /FM3050/ &
Man fügt der Eigenschaftenliste eine Eigenschaft hinzu, indem man auf den Button mit dem Pluszeichen und der Beschriftung "`Neu"' drückt. Die Checkbox mit der Beschriftung "`Analyse"' klickt man an. Man wählt im Parametereditor den Start der Analyse in der Toolbar aus.
 &
Die Eigenschaft erscheint grün. Die Eigenschaft wurde von CBMC überprüft. &
\centering . & \Checkmark
\\
\hline 
/T530/ &
/FM3010/ FM3020/ &
Man drückt auf den Button mit dem Pluszeichen und der Beschriftung "`Neu"'. 
 &
Eine neue Eigenschaft mit dem Name "`Eigenschaft 0"' erscheint in der Liste. &
\centering . & \Checkmark
\\
\hline



\end{tabular}
\end{table}