\begin{table}[]
\caption{Testfall 8.7 (Testfälle für den Parametereditor)}
\centering
	\begin{tabular}{| p{0.10\linewidth} | p{0.15\linewidth} | p{0.27\linewidth} |
	p{0.15\linewidth} | p{0.09\linewidth} | p{0.09\linewidth} |}
	\hline
	\textbf{Sub-Testfall} &
	\textbf{Abgedeckte Funktionalitäten} &
	\textbf{Beschreibung} &
	\textbf{Ergebnis} & \textbf{Jonas}
	(Windows 10 Version 1607) BEAST v1.4.18 &
	\textbf{Niels} (Linux Mint Cinnamon 3.0.7) Version 1.4.22
\\
\hline
/T610/ &
/FM4010/ /FM4020/ /FM4070/ &
Man versucht zunächst negative Zahlen oder 0 als Wähler, Kandidaten und Sitze anzugeben. Dann gibt man als Minimum größere Zahlen als das jeweilige Maximum und dann als Maximum kleinere Zahlen als das jeweilige Minimum an. Zuletzt gibt man sinnvolle Zahlen (alle größer als 0 und Minimum kleiner als Maximum an.
 &
Der Parametereditor setzt nach Eingabe der negativen Zahlen das entsprechende Feld auf den letzten validen Wert zurück. Nach Eingabe der größeren Minima und der kleineren Maxima wird der jeweilige andere Wert angepasst. Sinnvolle Zahlen werden angenommen. &
\centering \Checkmark & .
\\
\hline 
/T620/ &
/FM4020/ /FM4030/ &
Man hat ein Wahlverfahren und Eigenschaften geladen, sowie Parameter angegeben, deren Analyse länger als der zu testende Timeout dauert. Man gibt den Timeout im Parametereditor an. Man startet die Analyse.
 &
Die Überprüfung wird nach Ablauf der angegebenen Dauer abgebrochen. &
\centering \Checkmark & .
\\
\hline 




\end{tabular}
\end{table}